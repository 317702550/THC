\documentclass[]{article}
\usepackage{graphicx}
%opening
\usepackage{float}
\title{Pasos para compilar el archivo RaízCuadrada.tex}
\author{Maximiliano Proaño Bernal}

\begin{document}

\maketitle
Paso 1: Descargar los paquetes que hacían falta, estos son: Stackengine y ListOfItems.
\begin{figure}[H]
	\centering
	\includegraphics[width=0.5\linewidth]{"../THC/LaTeX/Captura 1"}
	\caption{}
	\label{fig:captura-1}
\end{figure}

\begin{figure}[H]
	\centering
	\includegraphics[width=0.5\linewidth]{"../THC/LaTeX/Captura 2"}
	\caption{}
	\label{fig:captura-2}
\end{figure}

Paso 2: Una vez hecho esto, se da click en compilar para saber si manda error nuevamente. Si sí, es necesario rectificar que se tienen todos los paquetes necesarios o si hace falta descargar alguno
\begin{figure}[H]
	\centering
	\includegraphics[width=0.5\linewidth]{"../THC/LaTeX/Captura 3"}
	\caption{}
	\label{fig:captura-3}
\end{figure}

\begin{figure}[H]
	\centering
	\includegraphics[width=0.5\linewidth]{"../THC/LaTeX/Captura 4"}
	\caption{}
	\label{fig:captura-4}
\end{figure}

Paso 3: Cuando el archivo se pueda compilar, guardar como PDF
\begin{figure}[H]
	\centering
	\includegraphics[width=0.5\linewidth]{"../THC/LaTeX/Captura 5"}
	\caption{}
	\label{fig:captura-5}
\end{figure}

\section{}

\end{document}
