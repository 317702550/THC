\documentclass{beamer}
\usetheme{default}
\usepackage{graphicx}
\usepackage{float}

\title{Presentación THC}
\author{Antonino Equihua Lombera, Maximiliano Proaño Bernal}
\date{10 de diciembre del 2019}
\institute{Facultad de Ciencias, UNAM}
\begin{document}
\begin{frame}[plain]
    \maketitle
\end{frame}
\begin{frame}{Pandas}
	Esta biblioteca nos ayuda a la manipulación de datos de tal manera que se puedan representar en forma de tablas
\end{frame}

\begin{frame}{Pandas}
	
\begin{figure}
	\centering
	\includegraphics[width=1.0\linewidth]{"../../Pictures/Captura Pandas"}
	\caption{}
	\label{fig:captura-pandas}
\end{figure}

\end{frame}

\begin{frame}{Kivy}
	Kivy es una biblioteca de código abierto de Python que se utiliza para crear aplicaciones móviles y también aplicaciones con una interfaz de usuario natural, se puede ejecutar Android, iOS, GNU/Linux, OS X, and Windows.
	
\end{frame}

\begin{frame}{Matplotlib}
	\textbf{Histograms:}\\ Este módulo nos permite mostrar la información estadísticamente en forma de histograma, el cual es una representación de la frecuencia de valores utilizando barras. Para esto, el módulo utiliza el concepto de "bin" como el número de divisiones para representar los datos.
\end{frame}

\begin{frame}{Matplotlib}
	
\begin{figure}
	\centering
	\includegraphics[width=1.0\linewidth]{"../../Pictures/Captura MHistograms"}
	\caption{}
	\label{fig:captura-mhistograms}
\end{figure}

\end{frame}

\begin{frame}{Matplotlib}
	\textbf{Contour:}\\ Con este módulo es posible representar la superficie de un espacio tridimensional en un plano bidimensional. La forma en la que se muestra dicha superficie es a base de colores. Por lo general se usan diferentes gradianes del mismo color para los "niveles" de la superficie, es decir: si el punto más alto se colorea de blanco y el más bajo de azul marino, las alturas que están en medio se colorean con mezclas de blanco y azul
\end{frame}

\begin{frame}{Matplotlib}
\begin{figure}
	\centering
	\includegraphics[width=1.0\linewidth]{"../../Pictures/Captura Contour"}
	\caption{}
	\label{fig:captura-contour}
\end{figure}

\end{frame}

\begin{frame}{Matplotlib}
	\textbf{Streamplot:}\\ Se usa para representar un campo vectorial en un plano, lo cual, para poder hacerlo, emplea flechas para la dirección de los vectores, modificaciones de colores, grosor de la líneas, entre otras cosas.
\end{frame}

\begin{frame}{Matplotlib}
	\textbf{Streamplot:}\\ Algunos de los parámetros que utiliza son:\\
	*x,y: Arreglos unidimensionales. Hace una cuadrícula\\
	*u,v: Arreglos bidimensionales. \\
	*Densidad: Controla la cercanía entre los vectores\\
	*Grosor de la línea: Modifica el grosor de dichos vectores\\
\end{frame}

\begin{frame}{Matplotlib}
	
\begin{figure}
	\centering
	\includegraphics[width=1.0\linewidth]{"../../Pictures/Captura Streamplot"}
	\caption{}
	\label{fig:captura-streamplot}
\end{figure}

\end{frame}

\begin{frame}{Matplotlib}
	\textbf{Interpolation:}\\ La interpolación es un tipo de estimación estadística que, a partir de valores relacionados y conocidos, se puede llegar a un resultado aproximado de valores desconocidos. Algunos de los métodos posibles son: “nearest”, “none”, “gaussian”, entre otros. Cada uno de ellos muestra una interpolación distinta
\end{frame}

\begin{frame}{Matplotlib}
\begin{figure}
	\centering
	\includegraphics[width=1.0\linewidth]{"../../Pictures/Captura interpolation"}
	\caption{}
	\label{fig:captura-interpolation}
\end{figure}
	
\end{frame}

\begin{frame}{Matplotlib}
	\textbf{Images:}\\ Se puede deducir del nombre que este módulo es capaz de mostrar una imagen guardada en los archivos, sin embargo, también es capaz de cambiar tonalidades de colores, escalas, entre otras cosas.
\end{frame}

\begin{frame}{Matplotlib}
	\begin{figure}
	\centering
	\includegraphics[width=1.0\linewidth]{"../../Pictures/Captura Images"}
	\caption{}
	\label{fig:captura-images}
\end{figure}

\end{frame}

\end{document}
