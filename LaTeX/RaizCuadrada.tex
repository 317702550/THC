\documentclass[letter,12pt]{article}
\usepackage[utf8]{inputenc}
\usepackage[spanish]{babel}
\pagenumbering{gobble}

%opening
\title{Una aproximación a la raíz cuadrada}
\author{Luis Enrique Serrano Gutiérrez}
\usepackage{tikz,stackengine}
\begin{document}

\maketitle


\begin{abstract}
Se presenta una forma de calcular la raíz cuadrada de un número positivo basada en la publicación ''Algoritmos Sencillos para Evaluar Funciones Elementales''escrita por el profesor Pablo Barrera Sánchez \cite{AlgoritmosPablo}.
\end{abstract}
\iffalse
 Es conocido el algoritmo para calcular la raíz cuadrada que requiere separar de dos en dos cifras de derecha a izquierda del punto decimal para buscar un número que multiplicado $y$ por si mismo sea lo más cercano al numero de a lo más dos cifras que se encuentre en el extremo izquierdo de la expresión, por el momento esta solución no es de nuestro interés.
\fi

\section*{La idea}
La raíz cuadrada de un numero $x$ está definida como un número $y$ tal que al ser multiplicado por si mismo se obtiene el número $x$, es decir:\\

$ x = y*y$\\

Recordemos que la fórmula para calcular el área de un cuadrado de lado l es:

 area del cuadrado de lado l $= l * l  $\\
 
 por lo que es posible hacer una interpretación geométrica de la raíz cuadrada del número $x$ como los lados de un cuadrado de área $x$.\\
 \pagebreak
\section*{El procedimiento}
Partiendo de esta interpretación el objetivo es calcular el valor $y$ a partir de la información inicial disponible, el valor de $x$. La primer pregunta por responder es: ¿A partir del valor $x$ puedo construir una figura de área $x$ de una forma sencilla?, la respuesta es sí, pues un rectángulo de altura 1 y base $x$ tiene área $x$.\\\\
%\enlargethispage{\baselineskip} 
\begin{center}
   { \begin{tikzpicture}
        \draw (0,0) -- (0,1) -- (5,1) -- (5,0) -- (0,0);
        \filldraw[black] (2.3,.2) circle (0pt)
        node[anchor=west] {x};        
        \filldraw[black] (5,.5) circle (0pt)
        node[anchor=west] {1};
    \end{tikzpicture}}
\end{center}
Una vez que se ha establecido un punto de partida, el rectángulo de base $b = x$ y altura $h = 1$ cuya área es $x$ (que se calcula con la fórmula $a = b * h$), se debe hacer la pregunta ¿cómo se deben modificar la base y la altura de ese rectángulo para que el rectángulo sea más parecido a un cuadrado, es decir, para que los lados sean más parecidos a la raíz cuadrada de $x$?. \\

Una forma de hacer esté cambio es cambiar la base por el promedio entre la base $b$ y la altura $h$:
$b$ = $(b+h)/2$\\

Para que el área $x$ sin cambios se debe de redefinir $h$ de la siguiente forma: $h = x/b$, de esta forma se mantiene la relación inicial en al que $x = b*h$

El nuevo rectángulo que se forma mantiene la misma área $x$ pero es más parecido a un cuadrado.\\

Con cada repetición de éste procedimiento en nuevo rectángulo que se obtiene se parecerá cada vez más a un cuadrado, y en consecuencia la longitud de cada uno de los lados se acercará cada vez más a la raiz cuadrada de $x$.
 \pagebreak
\section*{Ejemplos}
Para $x = 16$
\begin{table}[h!]
	\centering
	\begin{tabular}{ |c|c|c| } 
	  \multicolumn{3}{c}{$x = 16$}\\
	  \hline
	 	rectángulo & b & h \\
	  \hline
		1& 16	& 1\\
		2& 8,5000	& 1,8824\\
		3& 5,1912	& 3,0822\\
		4& 4,1367	& 3,8679\\
		5& 4,0023	& 3,9977\\
		6& 4,0000	& 4,0000\\
	 \hline
	\end{tabular}
\caption{Rectángulos para $x$ = 16 }
\label{table:1}
\end{table}

\begin{table}[h!]
	\centering
	\begin{tabular}{ |c|c|c| } 
	  \multicolumn{3}{c}{$x = 4$}\\
	  \hline
	 	rectángulo & b & h \\  
	  \hline
		  1& 4&	1\\ 
		  2& 2,5000&	1,6000\\
		  3& 2,0500&	1,9512\\
		  4& 2,0006&	1,9994\\
		  5& 2,0000&	2,0000\\
	 \hline
	\end{tabular}
	\caption{Rectángulos para $x$ = 4 }
	\label{table:2}
\end{table}
Esta forma de calcular la raíz cuadrada nos deja algunas preguntas, a saber:
\begin{enumerate}
\item ¿ Siempre se obtiene el valor exacto de la raíz?
\item ¿ Hay un numero fijo de rectángulos que se tengan que calcular antes de obtener la raíz?

\end{enumerate}


\begin{thebibliography}{0}
 \bibitem[PBS1996]{AlgoritmosPablo}Barrera Sánchez Pablo, Algoritmos Sencillos para Evaluar Funciones Elementales, 1996.
\end{thebibliography}



\end{document}
