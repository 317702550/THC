\documentclass[]{article}
\usepackage{float}
\usepackage{graphicx}
%opening
\title{Pasos para compilar el archivo Seno.tex}
\author{Maximiliano Proaño Bernal}

\begin{document}
\maketitle
Paso 1: Descargar el paquete que hacía falta: pgfplots
\begin{figure}[H]
	\centering
	\includegraphics[width=0.5\linewidth]{"Imagenes/Captura 1 Seno"}
	\caption{}
	\label{fig:captura-1-seno}
\end{figure}
\begin{figure}[H]`
	\centering
	\includegraphics[width=0.5\linewidth]{"Imagenes/Captura 2 Seno"}
	\caption{}
	\label{fig:captura-2-seno}
\end{figure}
Paso 2:Comentar la línea que citaba al programa en Python con el código
\begin{figure}[H]
	\centering
	\includegraphics[width=0.5\linewidth]{"Imagenes/Captura 3 Seno"}
	\caption{}
	\label{fig:captura-3-seno}
\end{figure}
\begin{figure}[H]
	\centering
	\includegraphics[width=0.5\linewidth]{"Imagenes/Captura 5 Seno"}
	\caption{}
	\label{fig:captura-5-seno}
\end{figure}


\section{}

\end{document}
