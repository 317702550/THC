\documentclass[]{article}
\usepackage{graphicx}
\usepackage{float}

%opening
\title{Reporte del archivo Seno.tex}
\author{Maximiliano Proaño Bernal}
\date{November 11, 2019}
\begin{document}
\maketitle
Cuando concluí la práctica pude notar que, analizando cómo se llevó a cabo la actividad, es evidente que tenía muchas similitudes con la práctica de Raíz Cuadrada. A pesar de que era un diferente concepto, el procedimiento y los pasos a seguir fueron muy similares.\\

Al empezar escribiendo el código del problema no tuve mayor complicación. Sin embargo, en las últimas líneas noté que algo no funcionaba como debía de hacerlo. Tras un pequeño análisis de mi código, observe que tenía una confución con las variables que usé. Afortunadamente pude corregir este error inmediatamente
\begin{figure}[H]
	\centering
	\includegraphics[width=1.0\linewidth]{"Imagenes/Captura 6 Seno"}
	\caption{}
	\label{fig:captura-6-seno}
\end{figure}
Una vez resuelto el problema con el archivo "ASEFE PBS Seno.py" no fue difícil implementar la misma idea al archivo "ASEFE PBS FuncSeno.py".
\begin{figure}[H]
	\centering
	\includegraphics[width=1.0\linewidth]{"Imagenes/Captura 7 Seno"}
	\caption{}
	\label{fig:captura-7-seno}
\end{figure}
En la figura podemos observar que el error de la función "seno" (la que debíamos hacer nosotros) con la función "sin" (la que ofrece Python) es increíblemente pequeño, lo que implica que la función "seno" realiza de manera casi perfecta su propósito.\\

Para concluir, se hacen unas preguntas al final del archivo Seno.tex: 1. ¿El valor de la potencia 5 es adecuado para todos los ángulos? 
2. ¿Entre mayor sea n la precisión incrementará?

La respuesta a la pregunta 1 es: Sí. No importa cuaĺ sea el ángulo en cuestión, la fórmula utilizada para calcular la cifra no tiene falla.

La respuesta a la pregunta 2 es: Sí. Este es un proceso que, al tener valores más grandes de n, el ángulo inicial será muy parecido a su seno, lo que incrementa la precisión del procedimiento.\\

Finalmente, quiero agregar que, a pesar de ser una práctica parecida a la anterior, sigue siendo un excelente ejericio para estudiantes principiantes en programación.
\section{}

\end{document}
